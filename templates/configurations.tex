
%------------------------------------- Fonts and Colors ---------------------
%\usepackage{xcolor}  %% already included by beamer

\usepackage{color}

\definecolor{midnightblue}{rgb}{0.2,0.2,0.7}

\definecolor{dkblue}{rgb}{0,0,0.6}
\definecolor{dkgreen}{rgb}{0,0.6,0}
\definecolor{dkred}{rgb}{0.6,0,0}

\definecolor{listinggray}{gray}{0.92}

\definecolor{headcolor}{rgb}{0.25,0.44,0.63}  % #4070a0 as in html.css

\usepackage{comment}

\usepackage{hyperref}

\usepackage{verbatim}

\usepackage{caption}

\usepackage{listings} % display code on slides; don't forget [fragile] option after \begin{frame}
\usepackage{lstlisting-llvm}

% Opçõess de listing usados para o codigo fonte
% Ref: http://en.wikibooks.org/wiki/LaTeX/Packages/Listings
\lstset{ %
language=Java,                  % choose the language of the code
basicstyle=\footnotesize,       % the size of the fonts that are used for the code
%basicstyle=\ttfamily,
stringstyle=\ttfamily\color[rgb]{0.16,0.16,0.16},
numbers=left,                   % where to put the line-numbers
numberstyle=\footnotesize,      % the size of the fonts that are used for the line-numbers
stepnumber=1,                   % the step between two line-numbers. If it's 1 each line will be numbered
numbersep=2pt,                  % how far the line-numbers are from the code
showspaces=false,               % show spaces adding particular underscores
showstringspaces=false,         % underline spaces within strings
showtabs=true,					% show tabs within strings adding particular underscores
frame=single,	                % adds a frame around the code
framerule=0.6pt,
tabsize=2,	                	% sets default tabsize to 2 spaces
%keepspaces,					% set one line at code final.
% extendedchars=true,
inputencoding=utf8,
extendedchars=true,
literate={á}{{\'a}}1 {ã}{{\~a}}1 {é}{{\'e}}1 {ç}{{\c{c}}}1,
captionpos=b,                   % sets the caption-position to bottom
breaklines=true,                % sets automatic line breaking
breakatwhitespace=false,        % sets if automatic breaks should only happen at whitespace
aboveskip=5pt,
upquote=true,
columns=fixed,
escapeinside={\%*}{*)},         % if you want to add a comment within your code
backgroundcolor=\color[rgb]{1.0,1.0,1.0}, % choose the background color.
rulecolor=\color[rgb]{0.8,0.8,0.8},
xleftmargin=10pt,
xrightmargin=10pt,
framexleftmargin=10pt,
framexrightmargin=10pt
}

%%% RAG----------------------------------------------------------------%
%-----------------------Style Definitions------------------------------%
\definecolor{javared}{rgb}{0.6,0,0} % for strings
\definecolor{javagreen}{rgb}{0.25,0.5,0.35} % comments
\definecolor{javapurple}{rgb}{0.5,0,0.35} % keywords
\definecolor{javadocblue}{rgb}{0.25,0.35,0.75} % javadoc

\definecolor{DarkBlue}{rgb}{0,0,0.61}
\definecolor{DarkGreen}{rgb}{0,0.4,0}
\definecolor{DarkRed}{rgb}{0.51,0,0}
\definecolor{DarkBlue2}{rgb}{0.25,0.51,0}

% Numeros.
\lstdefinestyle{mynumbers}{
	numbers=left,
	stepnumber=1,
	numbersep=4pt,
	numberstyle=\tiny\color{black}
}
% Text Code.
\lstdefinestyle{mytextcode}{
	basicstyle=\footnotesize,
	tabsize=2,
	showspaces=false,
	showstringspaces=false,
	extendedchars=true,
	breaklines=true
}
% Frame.
\lstdefinestyle{myframe}{
	backgroundcolor=\color{white},
	frame=trbl
}

% Formatar código automaticamente.
% http://tex.stackexchange.com/questions/166674/can-the-listings-package-automatically-indent-source-code
% Usar isso, carregando junto com a declaração do pacote.
% \usepackage[formats]{listings}
% ou:
\lstloadaspects{formats}

\lstdefineformat{C}
{
	\{=\newline\string\newline\indent,%
	\}=\newline\noindent\string\newline,%
	;=[\ ]\string\space,%
}
% C++ Style.
\lstdefinestyle{C++}{
	language=C++,
	format=C,
	style=mynumbers,
	style=mytextcode,
	style=myframe,
	keywordstyle=\color{black}\bfseries,
	stringstyle=\color{gray},
	commentstyle=\color[rgb]{0.439,0.50,0.56}, % rgb: 112, 128, 144 / 255 https://www.techfak.uni-bielefeld.de/~walter/misc/colorRGB.pdf
	morecomment=[s]{/*}{*/},
	morecomment=[l]{//},
	otherkeywords={dim3},
  	emph={ \_\_device\_\_, \_\_global\_\_, \_\_shared\_\_, \_\_host\_\_, \_\_constant\_\_},
	emphstyle=\color{DarkBlue}\bfseries,
	emph={[2] printf, scanf, \#include, \#define, \#pragma, \#typedef},
	emphstyle=[2]\color{DarkGreen},
}

% C Style.
\lstdefinestyle{C}{
	language=C,
	format=C,
	style=mynumbers,
	style=mytextcode,
	style=myframe,
	basicstyle=\ttfamily,
	keywordstyle=\color{javapurple}\bfseries,
  stringstyle=\color{gray}\bfseries,
  commentstyle=\color[rgb]{0.439,0.50,0.56}, % rgb: 112, 128, 144 / 255 https://www.techfak.uni-bielefeld.de/~walter/misc/colorRGB.pdf
  	% morecomment=[s][\color{lightgray}]{/*}{*/},
  	% morecomment=[l][\color{lightgray}]{//},
  	morecomment=[s]{/*}{*/}, % está usando a cor definida no commentstyle.
  	morecomment=[l]{//},
  	otherkeywords={dim3, \#define, \#pragma, \#typedef},
  	emph={ \_\_device\_\_, \_\_global\_\_, \_\_shared\_\_, \_\_host\_\_, \_\_constant\_\_},
  	emphstyle=\color{DarkBlue}\bfseries,
  	emph={[2] printf, scanf, \#include},
  	emphstyle=[2]\color{DarkGreen},
  	emph={[3] omp, oac},
  	emphstyle=[3]\color{DarkRed},
  	emph={[4] single, parallel, runtime},
  	emphstyle=[4]\color{DarkBlue2},
	backgroundcolor={},
	identifierstyle=\color{black}	
}
% Bash Style.
\lstdefinestyle{bash}{
	language=bash,
	style=mynumbers,
	style=mytextcode,
	style=myframe,
	backgroundcolor={},
	frame=single,
	commentstyle=\color[rgb]{0.439,0.50,0.56}, % rgb: 112, 128, 144 / 255 https://www.techfak.uni-bielefeld.de/~walter/misc/colorRGB.pdf
	%morecomment=[l]{#\},
	basicstyle=\scriptsize\ttfamily
}
% Python Style.
\lstdefinestyle{python}{
	language=python,
	style=mynumbers,
	style=mytextcode,
	style=myframe,
	backgroundcolor={}
}
% Java Style.
\lstdefinestyle{java}{
	language=java,
	style=mynumbers,
	style=mytextcode,
	style=myframe,
	backgroundcolor={},
	basicstyle=\ttfamily,
	keywordstyle=\color{javapurple}\bfseries,
	stringstyle=\color{javared},
	commentstyle=\color{javagreen},
	morecomment=[s][\color{javadocblue}]{/**}{*/}
}
% ASM Style.
\lstdefinestyle{asm}{
  %belowcaptionskip=1\baselineskip,
  %xleftmargin=\parindent,
  language=[x86masm]Assembler,
  style=mynumbers,
  style=mytextcode,
  style=myframe,
  backgroundcolor={},
  frame=single,
  basicstyle=\scriptsize\ttfamily,
  keywordstyle=\color{javapurple}\bfseries,
  stringstyle=\color{gray}\bfseries,
  commentstyle=\itshape\color{red!40!gray},
  morecomment=[l]{;},
  identifierstyle=\color{black},
  otherkeywords={movl, leaq, movq, subq, jmp, jg, pushq, popq, addl, cmpl, movss, cmpb, idivl, cltd},
  emph={eax, ebx, ecx, edx, esi, edi, ebp, eip, esp, r8d, r9d, r10d, r11d, r12d, r13d, r14d, r15d},
  emphstyle=\color{DarkGreen}\bfseries,
  emph={[2]rax, rbx, rcx, rdx, rsi, rdi, rbp, rip, rsp, r8, r9, r10, r11, r12, r13, r14, r15},
  emphstyle=[2]\color{DarkRed}\bfseries  
}

% Fortran Style.
\lstdefinestyle{fortran}{
  language=[90]Fortran,
  style=mynumbers,
  style=mytextcode,
  style=myframe,
  backgroundcolor={},
  frame=single,
  basicstyle=\footnotesize,
  commentstyle=\itshape\color{purple!40!black},
  morecomment=[l]{!\ }% Comment only with space after !
}

% LLVM Style.
\lstdefinestyle{llvm}{
	language=llvm,
	%inputencoding=utf8,
	style=mynumbers,
	style=mytextcode,
	style=myframe,
	backgroundcolor={},
	frame=single,
	basicstyle=\scriptsize\ttfamily,
  tabsize=4,
  %rulecolor=,
  upquote=true,
% aboveskip={1.5\baselineskip},
  columns=fixed,
  prebreak = \raisebox{0ex}[0ex][0ex]{\ensuremath{\hookleftarrow}},
  showtabs=false,
	%basicstyle=\scriptsize\upshape\ttfamily,
  identifierstyle=\ttfamily,
  keywordstyle=\ttfamily\bfseries\color[rgb]{0,0,0},
  %commentstyle=\ttfamily\color[rgb]{0.133,0.545,0.133},
  commentstyle=\ttfamily\color[rgb]{0.08,0.08,0.08},
  %stringstyle=\ttfamily\color[rgb]{0.627,0.126,0.941}
  stringstyle=\ttfamily\color[rgb]{0.16,0.16,0.16},
  morecomment = [l]{;},
    morestring=[b]", 
    sensitive = true,
    classoffset=0,
    morekeywords={
      define, declare, global, constant,
      internal, external, private,
      linkonce, linkonce_odr, weak, weak_odr, appending,
      common, extern_weak,
      thread_local, dllimport, dllexport,
      hidden, protected, default,
      except, deplibs,
      volatile, fastcc, coldcc, cc, ccc,
      x86_stdcallcc, x86_fastcallcc,
      ptx_kernel, ptx_device,
      signext, zeroext, inreg, sret, nounwind, noreturn,
      nocapture, byval, nest, readnone, readonly, noalias, uwtable,
      inlinehint, noinline, alwaysinline, optsize, ssp, sspreq,
      noredzone, noimplicitfloat, naked, alignstack,
      module, asm, align, tail, to,
      addrspace, section, alias, sideeffect, c, gc,
      target, datalayout, triple,
      blockaddress
    },
    classoffset=1, keywordstyle=\color{purple},
    morekeywords={
      fadd, sub, fsub, mul, fmul,
      sdiv, udiv, fdiv, srem, urem, frem,
      and, or, xor,
      icmp, fcmp,
      eq, ne, ugt, uge, ult, ule, sgt, sge, slt, sle,
      oeq, ogt, oge, olt, ole, one, ord, ueq, ugt, uge,
      ult, ule, une, uno,
      nuw, nsw, exact, inbounds,
      phi, call, select, shl, lshr, ashr, va_arg,
      trunc, zext, sext,
      fptrunc, fpext, fptoui, fptosi, uitofp, sitofp,
      ptrtoint, inttoptr, bitcast,
      ret, br, indirectbr, switch, invoke, unwind, unreachable,
      malloc, alloca, free, load, store, getelementptr,
      extractelement, insertelement, shufflevector,
      extractvalue, insertvalue,
    },
    alsoletter={\%},
    keywordsprefix={\%},
}

\lstdefinestyle{pascal}{
	language=pascal,
	format=pascal,
	style=mynumbers,
	style=mytextcode,
	style=myframe,
	keywordstyle=\color{black}\bfseries,
	stringstyle=\color{gray},
	commentstyle=\color[rgb]{0.439,0.50,0.56}, % rgb: 112, 128, 144 / 255 https://www.techfak.uni-bielefeld.de/~walter/misc/colorRGB.pdf
	morecomment=[s]{{}{}},
	morekeywords={
	}
	emph={ \_\_device\_\_, \_\_global\_\_, \_\_shared\_\_, \_\_host\_\_, \_\_constant\_\_},
	emphstyle=\color{DarkBlue}\bfseries,
	emph={[2] leia, escreva},
	emphstyle=[2]\color{DarkGreen},
}

% Adaptação para palavras até, senão, faça...
\newcommand\keywordstyle[1]{{\color{black}\bfseries{#1}}}%

%\usepackage{listings}
\lstdefinelanguage{tpp}{
    basicstyle=\ttfamily,
    sensitive=true,
    inputencoding=utf8,
    extendedchars=true,
    %alsoletter={_,ç,é,ã},
    %literate={á}{{\'a}}1 {ã}{{\~a}}1 {é}{{\'e}}1 {ç}{{\c{c}}}1,
    literate=%
        {é}{{\'{e}}}1
        {è}{{\`{e}}}1
        {ê}{{\^{e}}}1
        {ë}{{\¨{e}}}1
        {É}{{\'{E}}}1
        {Ê}{{\^{E}}}1
        {û}{{\^{u}}}1
        {ù}{{\`{u}}}1
        {â}{{\^{a}}}1
        {à}{{\`{a}}}1
        {á}{{\'{a}}}1
        {ã}{{\~{a}}}1
        {Á}{{\'{A}}}1
        {Â}{{\^{A}}}1
        {Ã}{{\~{A}}}1
        {ç}{{\c{c}}}1
        {Ç}{{\c{C}}}1
        {õ}{{\~{o}}}1
        {ó}{{\'{o}}}1
        {ô}{{\^{o}}}1
        {Õ}{{\~{O}}}1
        {Ó}{{\'{O}}}1
        {Ô}{{\^{O}}}1
        {î}{{\^{i}}}1
        {Î}{{\^{I}}}1
        {í}{{\'{i}}}1
        {Í}{{\~{Í}}}1
        {então}{{\keywordstyle{ent\~{a}o} }}5
        {senão}{{\keywordstyle{sen\~{a}o} }}5
        {faça}{{\keywordstyle{fa\c{c}a}}}4
        {até}{{\keywordstyle{at\'{e} }}}3,
    keywords = {se, até, então, senão, retorna, repita, fim, flutuante, inteiro, leia, escreva, principal, para, faça},
    otherkeywords={=, <, >, :, =, <=, >=, :=, <>},
    ndkeywords={leia,escreva},
    morecomment=[l]{//},
    morecomment=[s]{{\{}{\}}},
    morestring=[b]"
}


\lstdefinestyle{tpp}{
  language=tpp,
  %format=tpp,
  style=mynumbers,
  style=mytextcode,
  style=myframe,
  keywordstyle=\color{black}\bfseries,
  stringstyle=\color{gray},
  commentstyle=\color[rgb]{0.439,0.50,0.56}, % rgb: 112, 128, 144 / 255 https://www.techfak.uni-bielefeld.de/~walter/misc/colorRGB.pdf
  %morecomment=[s]{{}{}},
  %morekeywords={
  %  inteiro, fim 
  %}
  %emph={ },
  emphstyle=\color{DarkBlue}\bfseries,
  emph={[2] leia, escreva, retorna, principal},
  emphstyle=[2]\color{DarkGreen},
}

\lstdefinelanguage{C}{%
  language=C,
  format=C,
  style=mynumbers,
  style=mytextcode,
  style=myframe,
  basicstyle=\ttfamily,
  keywordstyle=\color{javapurple}\bfseries,
  stringstyle=\color{gray}\bfseries,
  commentstyle=\color[rgb]{0.439,0.50,0.56}, % rgb: 112, 128, 144 / 255 https://www.techfak.uni-bielefeld.de/~walter/misc/colorRGB.pdf
    % morecomment=[s][\color{lightgray}]{/*}{*/},
    % morecomment=[l][\color{lightgray}]{//},
    morecomment=[s]{/*}{*/}, % está usando a cor definida no commentstyle.
    morecomment=[l]{//},
    otherkeywords={dim3, \#define, \#pragma, \#typedef},
    emph={ \_\_device\_\_, \_\_global\_\_, \_\_shared\_\_, \_\_host\_\_, \_\_constant\_\_},
    emphstyle=\color{DarkBlue}\bfseries,
    emph={[2] printf, scanf, \#include},
    emphstyle=[2]\color{DarkGreen},
    emph={[3] omp, oac},
    emphstyle=[3]\color{DarkRed},
    emph={[4] single, parallel, runtime},
    emphstyle=[4]\color{DarkBlue2},
  backgroundcolor={},
  identifierstyle=\color{black}
}

%% Code
% \usepackage[table]{xcolor}
\usepackage{listings}
\lstdefinelanguage{Pseudo}{%
    sensitive=false,
    morekeywords={function, if, else, switch, case, while, for, True, False, and, or, xor, not, in, return, int},
    otherkeywords={=, <, >, :, ==, <=, >=, <-, :=, ..},
    morecomment=[l]{//},
    morecomment=[s]{{/*}{*/}},
    morestring=[b]"
}

\definecolor{greencomments}{rgb}{0,0.5,0}
\definecolor{bluekw}{rgb}{0,0,0.5}
\definecolor{redstrings}{rgb}{0.5,0,0}
\definecolor{graylinenumbers}{rgb}{0.5,0.5,0.5}
\lstset{%
    basicstyle=\footnotesize\ttfamily,
    columns=fullflexible,
    breaklines=true,
    breakatwhitespace,
    extendedchars=true,
    aboveskip=1em,
    belowskip=1em,
    % frame=single,
    numbers=left,
    xleftmargin=3em,
    numbersep=5pt,
    numberstyle=\tiny\color{graylinenumbers},
    rulecolor=\color{black},
    tabsize=2,
    commentstyle=\color{greencomments},
    keywordstyle=\color{bluekw}\bfseries,
    stringstyle=\color{redstrings},
    %language=Pseudo
    backgroundcolor=\color{white},
    frame=trbl,
    rulecolor=\color{gray}
}

\renewcommand{\lstlistingname}{Código}
\renewcommand{\figurename}{Figura}
\renewcommand{\tablename}{Tabela}

%%% RAG----------------------------------------------------------------%

%------------------------------------- Fonts and Colors ---------------------

\usepackage{todonotes}
\presetkeys{todonotes}{inline, linecolor=green!70!white, backgroundcolor=blue!30!white, bordercolor=black!50, color=blue!30, size=\normalsize}{}

% Solve the problem: ! Undefined control sequence.
% l.52   \passthrough{\lstinline!body!}.
% \newcommand{\passthrough}[1]{#1}
%\newcommand{\passthrough}{\shorthandoff{:}\passthroughtwo}
%\newcommand{\passthroughtwo}[1]{#1\shorthandon{:}}


\usepackage{graphicx}

% https://stackoverflow.com/questions/47962960/r-markdown-version-of-framesubtitle
% Tentativa de colocar blocos como subtítulos dos frames.
% \usepackage{environ}
% \RenewEnviron{block}[1]{\framesubtitle{#1}\BODY}


\usepackage[multiple]{footmisc}
